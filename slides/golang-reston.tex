\documentclass{beamer}
%\usetheme{metropolis}
%\usecolortheme{owl}
%\usecolortheme[snowy]{owl}
\usepackage{listings}
\usepackage{listings-golang}
\title{Own Your Own Dependencies}
\subtitle{Golang Reston}
\author{Michael Spiegel}
\date{January 16, 2018}

\lstset{
frame=single,
basicstyle=\footnotesize,
keywordstyle=\color{red},
showstringspaces=false,
stringstyle=\color{blue},
tabsize=4,
language=Golang
}

\hypersetup{
  colorlinks=true,
  urlcolor=blue
}

\begin{document}

\begin{frame}
\titlepage
\end{frame}

\begin{frame}
\frametitle{logging standards?}
\begin{itemize}
\item lots of prior discussions about loggers
\item \href{https://dave.cheney.net/2015/11/05/lets-talk-about-logging}{Let's Talk About Logging}
\item \href{https://go-talks.appspot.com/github.com/ChrisHines/talks/structured-logging/structured-logging.slide}{The Hunt for a Logger Interface}
\item logging levels should be actionable
\item level error goes to PagerDuty
\item level warning goes to non-immediate reporting
\item use structured logging to your advantage
\end{itemize}
\end{frame}

\begin{frame}
\frametitle{rs/zerolog to the rescue}
\begin{itemize}
\item \href{https://github.com/rs/zerolog}{github.com/rs/zerolog}
\item leveled logger
\item structured logger
\item zero allocation (or low allocation) logger
\item uses types not interfaces
\item 5055 lines of code
\item tradeoff: you can't inspect fields
\end{itemize}
\end{frame}

\begin{frame}[fragile]
\frametitle{contexts}
\begin{lstlisting}[basicstyle=\ttfamily\footnotesize]
package context

// The provided key must be comparable and should not
// be of type string or any other built-in type to
// avoid collisions between packages using context.
// Users of WithValue should define their own types
// for keys. To avoid allocating when assigning to
// an interface{}, context keys often have concrete
// type struct{}.
func WithValue(parent Context, key,
               val interface{}) Context {
  ...
  return &valueCtx{parent, key, val}
}
\end{lstlisting}
context $\rightarrow$ values $\rightarrow$ in $\rightarrow$ list \\
\pause
Actually \\
interface\{\} $\rightarrow$ interface\{\} $\rightarrow$ interface\{\}
$\rightarrow$ interface\{\}
\end{frame}

\begin{frame}[fragile]
\frametitle{type-safe contexts}
\begin{lstlisting}[basicstyle=\ttfamily\footnotesize]
package frame

type frameCtx struct{}

var frameKey frameCtx

type Frame struct {
	UUID   uuid.UUID
	Logger zerolog.Logger

	Foo string
	Bar bool
	Baz struct {
		A int
		B byte
		C string
	}
}
\end{lstlisting}
\end{frame}


\begin{frame}[fragile]
\frametitle{type-safe contexts}
\begin{lstlisting}[basicstyle=\ttfamily\footnotesize]
package frame

func NewContext(ctx context.Context) context.Context {
	fr := new(Frame)
	fr.Logger = log.Logger
	return context.WithValue(ctx, frameKey, fr)
}

func FromContext(ctx context.Context) *Frame {
	fr := ctx.Value(frameKey)
	if fr == nil {
		return nil
	}
	return fr.(*Frame)
}
\end{lstlisting}
\end{frame}

\begin{frame}
\frametitle{error handling}
\begin{itemize}
\item generate http response codes
\item combine multiple errors
\item generate http response codes for combined errors
\end{itemize}
\end{frame}

\begin{frame}[fragile]
\frametitle{external errors}
\begin{lstlisting}[basicstyle=\ttfamily\footnotesize]
package exterror

type ExtError struct {
	Status int
	Err    error
}

func (e ExtError) Error() string {
	return e.Err.Error()
}

func Create(status int, err error) ExtError {
	return ExtError{Status: status, Err: err}
}
\end{lstlisting}
\end{frame}

\begin{frame}[fragile]
\frametitle{combining errors}
\begin{lstlisting}[basicstyle=\ttfamily\footnotesize]
package multierr

type Error []error

func Append(e1 error, e2 error) error {
	if isNil(e1) && isNil(e2) { return nil }
	if isNil(e1) { return e2 }
	if isNil(e2) { return e1 }
	switch e1 := e1.(type) {
	case Error:
		switch e2 := e2.(type) {
			case Error: return append(e1, e2...)
			default: return append(e1, e2)
		}
	default:
		...
	}
}
\end{lstlisting}
\begin{itemize}
\item \href{https://github.com/jonbodner/multierr}{github.com/jonbodner/multierr}
\item 62 lines of code
\end{itemize}
\end{frame}

\begin{frame}[fragile]
\frametitle{combining external errors}
\begin{lstlisting}[basicstyle=\ttfamily\footnotesize]

// Convert generates an ExtError from
// a non-nil error input
func Convert(err error) ExtError {
	switch err := err.(type) {
	case ExtError:
		return err
	case multierr.Error:
		return ExtError{
			Status: convertMultiErr(err),
			Err: err
		}
	default:
		return ExtError{Status: 500, Err: err}
	}
}
\end{lstlisting}
\end{frame}

\begin{frame}[fragile]
\frametitle{combining external errors}
\begin{lstlisting}[basicstyle=\ttfamily\footnotesize]
func convertMultiErr(errs multierr.Error) int {
	if !allExtError(errs) {
		return 500
	} else if allEqualStatus(errs) {
		return errs[0].(ExtError).Status
	} else if allRangeStatus(errs, 400, 500) {
		return 400
	}
	return 500
}
\end{lstlisting}
\end{frame}

\begin{frame}[fragile]
\frametitle{response handler}
\begin{lstlisting}[basicstyle=\ttfamily\footnotesize]
func Response(handle respHandle) httprouter.Handle {
	return func(w http.ResponseWriter,
		r *http.Request, p httprouter.Params) {
		msg, err := handle(r, p)
		if err != nil {
			HandleError(w, r, err)
		} else {
			HandleResult(w, r, msg)
		}
	}
}
\end{lstlisting}
\end{frame}

\begin{frame}[fragile]
\frametitle{error response handler}
\begin{lstlisting}[basicstyle=\ttfamily\footnotesize]
func HandleError(w http.ResponseWriter,
	r *http.Request, err error) {

	ctx := r.Context()
	fr := frame.FromContext(ctx)
	exterr := exterror.Convert(ctx, err)
	if exterr.Status < 500 {
		fr.Logger.Warn().
			Err(err).
			Int("status", exterr.Status).
			Msg("user error reported")
	} else {
		fr.Logger.Error().
			Err(err).
			Int("status", exterr.Status).
			Msg("server error reported")
	}
	...
}
\end{lstlisting}
\end{frame}

\end{document}
